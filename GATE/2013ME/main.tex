\documentclass[journal,12pt,onecolumn]{IEEEtran}

% Import necessary packages
\usepackage{cite}
\usepackage{amsmath,amssymb,amsfonts,amsthm}
\usepackage{algorithmic}
\usepackage{graphicx}
\usepackage{textcomp}
\usepackage{xcolor}
\usepackage{txfonts}
\usepackage{listings}
\usepackage{enumitem}
\usepackage{mathtools}
\usepackage{gensymb}
\usepackage{comment}
\usepackage[breaklinks=true]{hyperref}
\usepackage{tkz-euclide} 
\usepackage{listings}
\usepackage{gvv}                                        
\usepackage[latin1]{inputenc}                                
\usepackage{color}                                            
\usepackage{array}                                            
\usepackage{longtable}                                       
\usepackage{calc}                                             
\usepackage{multirow}                                         
\usepackage{hhline}                                           
\usepackage{ifthen}                                           
\usepackage{lscape}
\usepackage{tabularx}
\usepackage{array}
\usepackage{float}
\usepackage{multicol} % Add the multicol package

% New theorem declarations
\newtheorem{theorem}{Theorem}[section]
\newtheorem{problem}{Problem}
\newtheorem{proposition}{Proposition}[section]
\newtheorem{lemma}{Lemma}[section]
\newtheorem{corollary}[theorem]{Corollary}
\newtheorem{example}{Example}[section]
\newtheorem{definition}[problem]{Definition}

% Custom command definitions
\newcommand{\BEQA}{\begin{eqnarray}}
\newcommand{\EEQA}{\end{eqnarray}}
\newcommand{\define}{\stackrel{\triangle}{=}}

\theoremstyle{remark}
\newtheorem{rem}{Remark}

% Document begins here
\begin{document}
\bibliographystyle{IEEEtran}
\vspace{3cm}

% Title of the document
\title{ASSIGNMENT 4}
\author{EE24BTECH11011 - PRANAY}
\maketitle

\bigskip

% Custom figure and table numbering
\renewcommand{\thefigure}{\theenumi}
\renewcommand{\thetable}{\theenumi}
   \section{\textbf{Linked Answer Questions}}
   \subsection{\textbf{Statement for Linked Answer Questions 52 and 53}}
   In the orthogonal turning of a bar of $100 mm$ diameter with a feed of $0.25 mm/rev$,depth of cut of $4 mm$ and cutting velocity of $90 m/min$,it is observed that the main (tangential) cutting force is perpendicular to the friction force acting at the chip-tool interface.The main (tangential) cutting force is $1500 N$.\\
\begin{enumerate}\setcounter{enumi}{52}
   \item The orthogonal rake angle of the cutting tool in degree is
   \begin{multicols}{4}
   \begin{enumerate}
       \item zero
       \item $3.58$
       \item $5$
       \item $7.16$
   \end{enumerate}
   \end{multicols}
   \item The normal force acting at chip-tool interface in $N$ is
   \begin{multicols}{4}
       \begin{enumerate}
           \item $1000$
           \item $1500$
           \item $2000$
           \item $2500$
       \end{enumerate}
   \end{multicols}
\end{enumerate}
\subsection{\textbf{Statement for Linked Answer Questions 54 and 55:}}
In a simple Brayton cycle, the pressure ratio is $8$ and temperatures at the entrance of compressor and
turbine are $300$ K and $1400$ K, respectively. Both compressor and gas turbine have isentropic efficiencies
equal to $0.8$. For the gas, assume a constant value of $c_p$ (specific heat at constant pressure) equal to $1 kJ/kgK$
and ratio of specific heats as $1.4$. Neglect changes in kinetic and potential energies.\\
\begin{enumerate}\setcounter{enumi}{53}
    \item The power required by compressor in $kW/kg$ of gas flow rate is
    \begin{multicols}{4}
        \begin{enumerate}
            \item $194.7$
            \item $243.4$
            \item $304.6$
            \item $378.5$
        \end{enumerate}
    \end{multicols}
    \item The thermal efficiency of the cycle in percentage $\brak{\%}$ is
    \begin{multicols}{4}
        \begin{enumerate}
            \item $24.8$
            \item $38.6$
            \item $44.8$
            \item $53.1$
        \end{enumerate}
    \end{multicols}
\end{enumerate}
\section{\textbf{General Aptitude Questions}}
\subsection{\textbf{Q.56 to Q.60 carry one mark each}}
\begin{enumerate}\setcounter{enumi}{55}
    \item Complete the sentence:
Universalism is to particularism as diffuseness is \rule{1cm}{0.15mm}
\begin{multicols}{4}
    \begin{enumerate}
        \item specificity
        \item neutrality
        \item generality
        \item adaptation
    \end{enumerate}
\end{multicols}
\item Were you a bird, \rule{1cm}{0.15mm} in the sky.
\begin{multicols}{4}
    \begin{enumerate}
        \item would fly
        \item shall fly
        \item should fly
        \item shall have flown
    \end{enumerate}
\end{multicols}
\item Which one of the following options is the closest in meaning to the word given below?
\begin{multicols}{4}
    \begin{enumerate}
        \item Highest
        \item Lowest
        \item Medium
        \item Integration
    \end{enumerate}
\end{multicols}
\item Choose the grammatically \textbf{INCORRECT} sentence:
\begin{enumerate}
    \item He is of Asian origin
    \item They belonged to Africa
    \item She is an European
    \item They migrated from India to Australia
\end{enumerate}
\item What will be the maximum sum of $44, 42, 40, \dots $?
\begin{multicols}{4}
    \begin{enumerate}
        \item $502$
        \item $504$
        \item $506$
        \item $500$
    \end{enumerate}
\end{multicols}
\end{enumerate}
\subsection{\textbf{Q.61 to Q.65 carry two marks each}}
\begin{enumerate}\setcounter{enumi}{60}
    \item Out of all the $2$-digit integers between $1$ and $100$, a $2$-digit number has to be selected at random. What is the probability that the selected number is not divisible by $7$?
    \begin{multicols}{4}
        \begin{enumerate}
            \item $\frac{13}{90}$
            \item $\frac{12}{90}$
            \item $\frac{78}{90}$
            \item $\frac{77}{90}$
        \end{enumerate}
    \end{multicols}
    \item A tourist covers half of his journey by train at $60$ km/h, half of the remainder by bus at $30$ km/h and
the rest by cycle at $10$ km/h. The average speed of the tourist in km/h during his entire journey is
\begin{multicols}{4}
    \begin{enumerate}
        \item $36$
        \item $30$
        \item $24$
        \item $18$
    \end{enumerate}
\end{multicols}
\item Find the sum of the expression
\begin{align}
    \frac{1}{\sqrt{1}+\sqrt{2}}+\frac{1}{\sqrt{2}+\sqrt{3}}+\frac{1}{\sqrt{3}+\sqrt{4}}+\dots + \frac{1}{\sqrt{80}+\sqrt{81}}
\end{align}
    \begin{enumerate}
        \item $7$ \item $8$ \item $9$ \item $10$
    \end{enumerate}
\item The current erection cost of a structure is Rs. $13,200$. If the labour wages per day increase by $\frac{1}{5}$ of
the current wages and the working hours decrease by $\frac{1}{24}$ of the current period, then the new cost
of erection in Rs. is
\begin{multicols}{4}
    \begin{enumerate}
        \item $16,500$
        \item $15,180$
        \item $11,000$
        \item $10,120$
    \end{enumerate}
\end{multicols}
\item After several defeats in wars, Robert Bruce went in exile and wanted to commit suicide. Just before
committing suicide, he came across a spider attempting tirelessly to have its net. Time and again,
the spider failed but that did not deter it to refrain from making attempts. Such attempts by the
spider made Bruce curious. Thus, Bruce started observing the near-impossible goal of the spider to
have the net. Ultimately, the spider succeeded in having its net despite several failures. Such act of
the spider encouraged Bruce not to commit suicide. And then, Bruce went back again and won
many a battle, and the rest is history.\\
Which of the following assertions is best supported by the above information?
\begin{enumerate}
    \item Failure is the pillar of success.
    \item Honesty is the best policy.
    \item Life begins and ends with adventures.
    \item No adversity justifies giving up hope.
\end{enumerate}
\end{enumerate}
\end{document}
