\documentclass[journal,12pt,onecolumn]{IEEEtran}

% Import necessary packages
\usepackage{cite}
\usepackage{amsmath,amssymb,amsfonts,amsthm}
\usepackage{algorithmic}
\usepackage{graphicx}
\usepackage{textcomp}
\usepackage{xcolor}
\usepackage{txfonts}
\usepackage{listings}
\usepackage{enumitem}
\usepackage{mathtools}
\usepackage{gensymb}
\usepackage{comment}
\usepackage[breaklinks=true]{hyperref}
\usepackage{tkz-euclide} 
\usepackage{listings}
\usepackage{gvv}                                        
\usepackage[latin1]{inputenc}                                
\usepackage{color}                                            
\usepackage{array}                                            
\usepackage{longtable}                                       
\usepackage{calc}                                             
\usepackage{multirow}                                         
\usepackage{hhline}                                           
\usepackage{ifthen}                                           
\usepackage{lscape}
\usepackage{tabularx}
\usepackage{array}
\usepackage{float}
\usepackage{multicol} % Add the multicol package
\usepackage{circuitikz}
\usepackage{textcomp}
% New theorem declarations
\newtheorem{theorem}{Theorem}[section]
\newtheorem{problem}{Problem}
\newtheorem{proposition}{Proposition}[section]
\newtheorem{lemma}{Lemma}[section]
\newtheorem{corollary}[theorem]{Corollary}
\newtheorem{example}{Example}[section]
\newtheorem{definition}[problem]{Definition}

% Custom command definitions
\newcommand{\BEQA}{\begin{eqnarray}}
\newcommand{\EEQA}{\end{eqnarray}}
\newcommand{\define}{\stackrel{\triangle}{=}}

\theoremstyle{remark}
\newtheorem{rem}{Remark}

% Document begins here
\begin{document}
\bibliographystyle{IEEEtran}
\vspace{3cm}

% Title of the document
\title{ASSIGNMENT 5}
\author{EE24BTECH11011 - PRANAY}
\maketitle

\bigskip

% Custom figure and table numbering
\renewcommand{\thefigure}{\theenumi}
\renewcommand{\thetable}{\theenumi}
   \begin{enumerate}\setcounter{enumi}{39}
       \item Using the Gauss-Seidel iteration method with the initial guess $\cbrak{x_1^{\brak{0}} = 3.5 , x_2^{\brak{0}} = 2.25,x_3^{\brak{0}} = 1.625}$,the second approximation $\cbrak{x_1^{\brak{2}},x_2^{\brak{2}},x_3^{\brak{2}}}$ for the solution to the system of equations
       \begin{align}
           2x_1 - x_2 = 7\\
           -x_1 +2x_2-x_3 =1\\
           -x_2+2x_2=1
       \end{align}
       is
       \begin{enumerate}
           \item $x_1^{\brak{2}}=5.3125, x_2^{\brak{2}}=4.4491, x_3^{\brak{2}} = 2.1563$\\
            \item $x_1^{\brak{2}}=5.3125, x_2^{\brak{2}}=4.3125, x_3^{\brak{2}} = 2.6563$\\
             \item $x_1^{\brak{2}}=5.3125, x_2^{\brak{2}}=4.4491, x_3^{\brak{2}} = 2.6563$\\
              \item $x_1^{\brak{2}}=5.4491, x_2^{\brak{2}}=4.4491, x_3^{\brak{2}} = 2.1563$\\
       \end{enumerate}
       \item The fourth order Runge-Kutta method given by
       \begin{align}
           u_{j+1} = u_j +\frac{h}{6}\sbrak{K_1+2K_2+2K_3+K_4}, j=0,1,2,\dots,
       \end{align}
       is used to solve the initial value problem $\frac{du}{dt}=u , u\brak{0}=\alpha$ .If $u\brak{1}=1$ is obtained by taking the step size  $h=1$,then the value of $K_4$ is \rule{3cm}{0.15mm}\\
       \item A particle $P$ of mass $m$ moves along the cycloid $x = \brak{\theta - \sin\theta}$ and $y = \brak{1+\cos\theta} , 0 \leq \theta \leq 2\pi$.Let $g$ denote the accelaration due to gravity.Neglecting the fractional force,the Lagrangian associated with the motion of particle $P$ is :
       \begin{enumerate}
           \item $m\brak{1-\cos\theta}\theta^{2} - mg\brak{1+\cos\theta} $
           \item $m\brak{1+\cos\theta}\theta^{2} + mg\brak{1+\cos\theta} $
           \item $m\brak{1+\cos\theta}\theta^{2} + mg\brak{1-\cos\theta} $
           \item $m\brak{1-\sin\theta}\theta^{2} - mg\brak{1+\cos\theta} $\\
       \end{enumerate}
       \item Suppose that $X$ is a population random variable with probability density function
       \begin{align}
           f\brak{x;\theta} = \begin{cases} \theta x^{\theta-1} , \text{ if } 0<x<1 \\ 0 \text{ otherwise } \end{cases}
       \end{align}
       where $\theta$ is a parameter .In order to test the null hypothesis $H_0 \colon \theta = 2$,against the alternative hypothesis $H_1 \colon \theta = 3$,the following test is used:Reject the null hypothesis if $X_1 \geq \frac{1}{2}$ and accept otherwise,where $X_1$ is a random sample of size $1$ drawn from the above population.Then the power of the test is \rule{3cm}{0.15 mm}
       \item Suppose that $X_1,X_2,\dots,X_3$ is a random sample of size $n$ from a population with probability density function
       \begin{align}
           f\brak{x;\theta} = \begin{cases}
               \frac{x}{\theta^2}e^{-\frac{x}{\theta}} \text{ if } x>0 \\ 0 \text{ otherwise }
           \end{cases}
       \end{align}
       where $\theta$ is a parameter such that $\theta > 0$.The maximum likelihood estimator of $\theta $ is
       \begin{multicols}{2}
       \begin{enumerate}
           \item $\frac{\sum_{i=1}^{n} X_i}{n}$\\
            \item $\frac{\sum_{i=1}^{n} X_i}{n-1}$
             \item $\frac{\sum_{i=1}^{n} X_i}{2n}$\\
              \item $\frac{2\sum_{i=1}^{n} X_i}{n}$
       \end{enumerate}
       \end{multicols}
       \item Let $\vec{F}$ be a vector field on $\mathbb{R}^2 / \cbrak{\brak{0,0}}$ by $\vec{F}\brak{x,y} = \frac{y}{x^2 + y^2}\hat{i} - \frac{x}{x^2 + y^2}\hat{j}$. Let $\gamma , \alpha \colon \sbrak{0,1} \rightarrow \mathbb{R}^2$ be defined by 
       \begin{align}
           \gamma \brak{t} = \brak{8 \cos{2 \pi t} , 17\sin{2\pi t}} \text{ and } \alpha\brak{t} = \brak{26\cos{2\pi t}, -10\sin{2 \pi t}}
       \end{align}
       If $3\int_{\alpha}\vec{F}\cdot d\vec{r} - 4\int_{\gamma}\vec{F}\cdot d\vec{r} = 2m\pi$ then $m$  is \rule{3cm}{0.15mm}\\
       \item Let $g \colon \mathbb{R}^3 \rightarrow \mathbb{R}$ be defined by $g\brak{x,y,z} = \brak{3y + 4z , 2x - 3z , x+ 3y}$ and let\begin{align} S = \cbrak{\brak{x,y,z} \in \mathbb{R}^3 \colon 0 \leq x \leq 1 , 0 \leq y \leq 1 , 0 \leq z \leq 1} \end{align} If
       \begin{align}
           \iiint \limits_{g\brak{s}}\brak{2x+y-2z}dxdydz = \alpha \iiint\limits_S z dx dy dz
       \end{align}
       Then $\alpha$ is \rule{3cm}{0.15mm}
       \item Let $T_1,T_2 \colon \mathbb{R}^5 \rightarrow \mathbb{R}^3$ be the linear transformations such that $rank\brak{T_1} = 3$ and $nullity\brak{T_2} = 3$ . Let $T_3 \colon \mathbb{R}^3 \rightarrow \mathbb{R}^3$ be a linear transformation such that $T_3 \circ T_1 = T_2$ Then $rank\brak{T_3}$ is \rule{3cm}{0.15mm}\\
       \item Let $\mathbb{F}_3$ be the field of 3 elements and let $\mathbb{F}_3 \times \mathbb{F}_3$ be a vector space of $\mathbb{F}_3$. Then the number of distinct linearly dependent sets of the form $\cbrak{u,v}$,where $u,v \in \mathbb{F}_3 \times \mathbb{F}_3 / \cbrak{\brak{0,0}}$ and $ u \neq v$ is \rule{3cm}{0.15mm}\\
       \item Let $\mathbb{R}_{125}$ be a field of $125$ elements. Then number of non-zero elements $\alpha \in \mathbb{F}_{125}$ such that $\alpha^5 = \alpha$ is \rule{3cm}{0.15mm}
       \item The value of $\iint \limits_R x y dx dy$, where $R$ is the region in the first quadrant bounded  by the curves $y = x^2 , y+x=2 \text{ and } x =0$ is \rule{3cm}{0.15mm}\\
       \item Consider the heat equation 
       \begin{align}
           \frac{\partial u}{\partial t} = \frac{\partial ^2 u}{\partial x^2} , 0<x<\pi , t>0
       \end{align}
       with the boundary conditions $u\brak{0,t} = 0 , u\brak{\pi , t} = 0$ for $t>0$ and the intial condition $u\brak{x,0} = \sin{x}$ . Then $u\brak{\frac{\pi}{2},1}$ is \rule{3cm}{0.15mm}\\
       \item Consider the partial order in $\mathbb{R}^2$ given by the relation $\brak{x_1,y_1} < \brak{x_2,y_2}$ EITHER if $x_1 < x_2$ OR if $x_1 = x_2$ and $y_1 < y_2$. Then the order topology on $\mathbb{R}^2$ defined by the above order
       \begin{enumerate}
           \item $\sbrak{0,1} \times \cbrak{1}$ is compact but $\sbrak{0,1} \times \sbrak{0,1}$ is NOT compact
           \item $\sbrak{0,1} \times \sbrak{0,1}$ is compact but $\sbrak{0,1} \times \cbrak{1}$ is NOT compact
           \item both $\sbrak{0,1} \times \sbrak{0,1}$ and $\sbrak{0,1} \times \cbrak{1}$ are compact
           \item both $\sbrak{0,1} \times \sbrak{0,1}$ and $\sbrak{0,1} \times \cbrak{1}$ are NOT  compact
       \end{enumerate}
   \end{enumerate}
\end{document}
