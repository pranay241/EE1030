\documentclass[journal,12pt,onecolumn]{IEEEtran}

% Import necessary packages
\usepackage{cite}
\usepackage{amsmath,amssymb,amsfonts,amsthm}
\usepackage{algorithmic}
\usepackage{graphicx}
\usepackage{textcomp}
\usepackage{xcolor}
\usepackage{txfonts}
\usepackage{listings}
\usepackage{enumitem}
\usepackage{mathtools}
\usepackage{gensymb}
\usepackage{comment}
\usepackage[breaklinks=true]{hyperref}
\usepackage{tkz-euclide} 
\usepackage{listings}
\usepackage{gvv}                                        
\usepackage[latin1]{inputenc}                                
\usepackage{color}                                            
\usepackage{array}                                            
\usepackage{longtable}                                       
\usepackage{calc}                                             
\usepackage{multirow}                                         
\usepackage{hhline}                                           
\usepackage{ifthen}                                           
\usepackage{lscape}
\usepackage{tabularx}
\usepackage{array}
\usepackage{float}
\usepackage{multicol} % Add the multicol package

% New theorem declarations
\newtheorem{theorem}{Theorem}[section]
\newtheorem{problem}{Problem}
\newtheorem{proposition}{Proposition}[section]
\newtheorem{lemma}{Lemma}[section]
\newtheorem{corollary}[theorem]{Corollary}
\newtheorem{example}{Example}[section]
\newtheorem{definition}[problem]{Definition}

% Custom command definitions
\newcommand{\BEQA}{\begin{eqnarray}}
\newcommand{\EEQA}{\end{eqnarray}}
\newcommand{\define}{\stackrel{\triangle}{=}}

\theoremstyle{remark}
\newtheorem{rem}{Remark}

% Document begins here
\begin{document}
\bibliographystyle{IEEEtran}
\vspace{3cm}

% Title of the document
\title{ASSIGNMENT 2}
\author{EE24BTECH11011 - PRANAY}
\maketitle

\bigskip

% Custom figure and table numbering
\renewcommand{\thefigure}{\theenumi}
\renewcommand{\thetable}{\theenumi}
\begin{enumerate}\setcounter{enumi}{26}
	\item Let $I = \int {\frac{f\brak{z}}{\brak{{z-1}{z-2}}}}dz$,where $f\brak{x} = \sin{\frac{\pi z}{2}} + \cos{\frac{\pi z}{2}}$ and $C$ is the curve $\abs{z} = 3$
oriented anticlockwise.Then the value of $I$ is 
\begin{enumerate}
\item $4\pi i$
\item $0$
\item $-2\pi i$
\item $-4 \pi i$\\
\end{enumerate}
\item  Let $\sum_{-\infty}^{\infty}b_n z^n$ be the Laurent series expansion of the function $\frac{1}{z\sinh{z}} , 0<\abs{z}<\pi$ . Then which one of the following is correct?
\begin{multicols}{2}
\begin{enumerate}
    \item $b_{-2} = 1 ,b_0 = -\frac{1}{6} , b_2 = \frac{7}{360}$\\
    \item $b_{-3} = 1 ,b_{-1} = -\frac{1}{6} , b_1 = \frac{7}{360}$
    \item $b_{-2} = 0 ,b_0 = -\frac{1}{6} , b_2 = \frac{7}{360}$\\
    \item $b_{0} = 1 ,b_2 = -\frac{1}{6} , b_6 = \frac{7}{360}$
\end{enumerate}
\end{multicols}
\item Under the transformation $w = \sqrt{\frac{1-iz}{z-i}}$, the region $D = \cbrak{z \in \mathbb{C}\colon \abs{z}<1}$ is transformed to
\begin{enumerate}
    \item $\cbrak{z \in \mathbb{C}\colon 0<arg z<\pi}$
    \item $\cbrak{z \in \mathbb{C}\colon -\pi<arg z<0}$
    \item $\cbrak{z \in \mathbb{C}\colon 0<arg z<\frac{\pi}{2} \text{ or } \pi < arg z < \frac{3\pi}{2}}$
    \item $\cbrak{z \in \mathbb{C}\colon \frac{\pi}{2}<arg z<\pi \text{ or } \frac{3\pi}{2}< arg z < 2\pi}$\\
\end{enumerate}
  \item Let $y\brak{x}$ be the solution of the initial value problem 
  \begin{align}
      y^{\prime\prime\prime} -y^{\prime\prime} + 4y^\prime -4y =0 \hspace{0.4cm} ,y\brak{0}=y^\prime \brak{0} = 2  \hspace{0.4cm},  y^{\prime\prime}\brak{0}=0
  \end{align}
  Then the value of $y\brak{\frac{\pi}{2}}$
  \begin{multicols}{4}
  \begin{enumerate}
      \item $\frac{1}{5}\brak{4e^{\frac{\pi}{2}}-6}$
            \item $\frac{1}{5}\brak{6e^{\frac{\pi}{2}}-4}$
                  \item $\frac{1}{5}\brak{8e^{\frac{\pi}{2}}-2}$
                        \item $\frac{1}{5}\brak{8e^{\frac{\pi}{2}}+2}$\\
  \end{enumerate}
  \end{multicols}
  \item Let $y\brak{x}$ be the solution of the initial value problem 
  \begin{align}
      x^2y^{\prime\prime} + x y^{\prime} + y = x , \hspace{0.5cm} y\brak{1} = y^\prime \brak{1} = 1
  \end{align}
  Then the value of $y\brak{e^{\frac{\pi}{2}}}$ is
  \begin{multicols}{4}
      \begin{enumerate}
          \item $\frac{1}{2}\brak{1 - e^{\frac{\pi}{2}}}$
           \item $\frac{1}{2}\brak{1 + e^{\frac{\pi}{2}}}$
            \item $\frac{1}{2}+\frac{\pi}{4}$
             \item $\frac{1}{2} - \frac{\pi}{4}$\\
      \end{enumerate}
  \end{multicols}
  \item Let $T \colon P_3\sbrak{0,1} \rightarrow P_2\sbrak{0,1}$ be defined by $\brak{Tp}\brak{x} = p^{\prime\prime}\brak{x} + p^{\prime}\brak{x}$.Then the matrix representation of $T$ with respect to the bases $\cbrak{1,x,x^2,x^3}$ and $\cbrak{1,x,x^2}$ of $P_3\sbrak{0,1}$ and $P_2\sbrak{0,1}$ respectively is
  \begin{multicols}{4}
      \begin{enumerate}
          \item $\myvec{0 & 0 & 0 \\ 1 & 0 & 0 \\2 & 2 & 0\\0 & 6 & 3}$
          \item $\myvec{0 & 1 & 2 & 0 \\ 0 & 0 & 2 & 6\\0 & 0 & 0 & 3}$
         \item $\myvec{0 & 2& 1 & 0 \\ 6 & 2 & 0 & 0\\3 & 0 & 0 & 0}$
          \item $\myvec{0 & 0 & 0 \\ 0 & 0 & 1 \\0 & 2 & 2\\3 & 6 & 0}$
      \end{enumerate}
  \end{multicols}
  \item Let $T \colon \mathbb{R}^3 \rightarrow \mathbb{R}^3$ be a linear transformations defined by  $T\brak{x,y,z} = \brak{x+y,y+z,z+x}$ then the orthonormal basis for the range $T$ is
  \begin{multicols}{2}
      \begin{enumerate}
          \item $\cbrak{\brak{\frac{1}{\sqrt 2},\frac{1}{\sqrt 2},0}\brak{\frac{1}{\sqrt 3},-\frac{1}{\sqrt 3},\frac{1}{\sqrt 3}}}$
          \item $\cbrak{\brak{\frac{1}{\sqrt 2},-\frac{1}{\sqrt 2},0}\brak{\frac{1}{\sqrt 6},\frac{1}{\sqrt 6},\frac{2}{\sqrt 6}}}$
              \item $\cbrak{\brak{\frac{1}{\sqrt 2},\frac{1}{\sqrt 2},0}\brak{\frac{1}{\sqrt 6},-\frac{1}{\sqrt 6},-\frac{2}{\sqrt 6}}}$
          \item $\cbrak{\brak{\frac{1}{\sqrt 2},\frac{1}{\sqrt 2},0}\brak{\frac{1}{\sqrt 3},-\frac{1}{\sqrt 3},-\frac{1}{\sqrt 3}}}$\\
      \end{enumerate}
  \end{multicols}
  \item Consider the basis $\cbrak{u_1,u_2,u_3}$ of $\mathbb{R}^3$,where $u_1 = \brak{1,0,0}$ $u_2\brak{1,1,0}$ , $u_3\brak{1,1,1}$.Let $\cbrak{f_1,f_2,f_3}$ be the dual basis of $\cbrak{u_1,u_2,u_3}$ ad $f$ be a linear functional defined by $f\brak{a,b,c} = a+b+c , \brak{a,b,c}\in\mathbb{R}^3$.If $f = \alpha_1f_1+\alpha_2f_2+\alpha_3+f_3$,then $\brak{\alpha_1,\alpha_2,\alpha_3}$ is
  \begin{multicols}{4}
      \begin{enumerate}
          \item $\brak{1,2,3}$
          \item $\brak{1,3,2}$
          \item $\brak{2,3,1}$
          \item $\brak{3,2,1}$
      \end{enumerate}
  \end{multicols}
  \item The following table gives the cost matrix of a transportation problem
\begin{table}[h!]    
        \centering
        \begin{tabular}{|c|c|c|c|}
 \hline
 4 & 5 & 6 \\ \hline
  3 & 2 & 2 \\ \hline
  1 & 1 & 2 \\ \hline
\end{tabular}

    \end{table}
  The basic feasible solution given by $x_{11} =3 , x_{13}=1 , x_{23} =6,x_{31}=2,x_{32}=5$ is
  \begin{enumerate}
      \item degenerate and optimal
      \item optimal but not degenerate
      \item degenerate but not optimal
      \item neither degenerate nor optimal
  \end{enumerate}
  \item If $z^*$ is the optimal value of the linear programming problem
  \begin{align}
      \text{ Maximize } z = 5x_1 + 9x_2 + 4x_3
  \end{align}
  \begin{align}
      \text{subject to } x_1 + x_2 + x_3 =5 
  \end{align}
  \begin{align}
      4x_1 + 3x_2 + 2x_3 = 12
  \end{align}
  \begin{align}
      x_1,x_2,x_3\geq 0,
  \end{align}
  then 
  \begin{enumerate}
      \item $0\leq z^* 10$
       \item $10\leq z^* 20$
        \item $20\leq z^* 30$
         \item $30\leq z^* 40$\\
  \end{enumerate}
  \item Let $G_1$ be an abelian group of order 6 and $G_2 = S_3$.For $j = 1,2$,let $P_j$ be the statement "$G_j$ has an unique order of $2$".Then
  \begin{enumerate}
      \item both $P_1$ ad $P_2$ holds
      \item neither $P_1$ nor $P_2$ holds 
      \item $P_1$ holds but not $P_2$
      \item $P_2$ holds but not $P_1$
  \end{enumerate}
  \item Let $G$ be group of all symmetries of the square .Then the number of conjugate classes in $G$ is 
  \begin{multicols}{4}
      \begin{enumerate}
          \item $4$
          \item $5$
          \item $6$
          \item $7$
      \end{enumerate}
  \end{multicols}
  \item Consider the polynomial ring $\mathbf{Q}\sbrak{x}$.The ideal of $\mathbf{Q}\sbrak{x}$ generated by $x^2 -3 $ is
  \begin{enumerate}
      \item maximal but not prime
      \item prime but not maximal
      \item both maximal and prime
      \item neither maximal nor prime
  \end{enumerate}
\end{enumerate}
\end{document}
