\let\negmedspace\undefined
\let\negthickspace\undefined
\documentclass[journal]{IEEEtran}
\usepackage[a5paper, margin=10mm, onecolumn]{geometry}
%\usepackage{lmodern} % Ensure lmodern is loaded for pdflatex
\usepackage{tfrupee} % Include tfrupee package

\setlength{\headheight}{1cm} % Set the height of the header box
\setlength{\headsep}{0mm}     % Set the distance between the header box and the top of the text

\usepackage{gvv-book}
\usepackage{gvv}
\usepackage{cite}
\usepackage{amsmath,amssymb,amsfonts,amsthm}
\usepackage{algorithmic}
\usepackage{graphicx}
\usepackage{textcomp}
\usepackage{xcolor}
\usepackage{txfonts}
\usepackage{listings}
\usepackage{enumitem}
\usepackage{mathtools}
\usepackage{gensymb}
\usepackage{comment}
\usepackage[breaklinks=true]{hyperref}
\usepackage{tkz-euclide} 
\usepackage{listings}
% \usepackage{gvv}                                        
\def\inputGnumericTable{}                                 
\usepackage[latin1]{inputenc}                                
\usepackage{color}                                            
\usepackage{array}                                            
\usepackage{longtable}                                       
\usepackage{calc}                                             
\usepackage{multirow}                                         
\usepackage{hhline}                                           
\usepackage{ifthen}                                           
\usepackage{lscape}
\begin{document}

\bibliographystyle{IEEEtran}
\vspace{3cm}

\title{MAINS}
\author{EE24BTECH11011-B.PRANAY KUMAR
}
 \maketitle
% \newpage
% \bigskip
{\let\newpage\relax\maketitle}

\renewcommand{\thefigure}{\theenumi}
\renewcommand{\thetable}{\theenumi}
\setlength{\intextsep}{10pt} % Space between text and floats


\numberwithin{equation}{enumi}
\numberwithin{figure}{enumi}
\renewcommand{\thetable}{\theenumi}
\begin{enumerate}\setcounter{enumi}{15}
    
    \item If $\frac{dy}{dx} = \frac{xy}{x^2+y^2}$ and $y\brak{1}=1$, then a value of $x$ satisfying $y\brak{x}=e$ is:
    \begin{multicols}{2} % Use multicol to divide options into two columns
    \begin{enumerate}
        \item $\sqrt{3} e$\\
        \item $\frac{1}{2} \sqrt{3} e$
        \item $\sqrt{2}e$\\
        \item $\frac{e}{\sqrt{2}}$
    \end{enumerate}
    \end{multicols}

    % Second question
    \item If one end of focal chord $AB$ of the parabola $y^2 = 8x$ is at $A\brak{\frac{1}{2},-2}$, then the equation of the tangent at $B$ is:
    \begin{multicols}{2}
    \begin{enumerate}
        \item $x+2y+8=0$
        \item $2x-y-24=0$
        \item $x-2y+8=0$
        \item $2x+y-24=0$
    \end{enumerate}
    \end{multicols}

    % Third question
    \item Let $a_n$ be the $n^{th}$ term of a G.P. of positive terms. If $\sum_{n=1}^{100} a_{2n+1} = 200$ and $\sum_{n=1}^{100}a_{2n} = 100$, then $\sum_{n=1}^{200}a_n$ is equal to:
    \begin{multicols}{2}
    \begin{enumerate}
        \item $300$\\
        \item $175$
        \item $225$\\
        \item $150$
    \end{enumerate}
    \end{multicols}

\item A random variable $X$ has the following probability distribution.
    \begin{table}[h!]    
        \centering
        \begin{tabular}{|c|c|c|c|}
 \hline
 4 & 5 & 6 \\ \hline
  3 & 2 & 2 \\ \hline
  1 & 1 & 2 \\ \hline
\end{tabular}

    \end{table}
    Then $P\brak{X>2}$ is:
    \begin{multicols}{2}
    \begin{enumerate}
        \item $\frac{7}{12}$\\
        \item $\frac{23}{26}$
        \item $\frac{1}{36}$\\
        \item $\frac{1}{6}$
    \end{enumerate}
    \end{multicols}

    % Fifth question
    \item If $\int \frac{d\theta}{\cos^2 \theta (\tan 2\theta + \sec 2 \theta)} = \lambda \tan \theta + 2 \log_e \abs {f\brak{\theta}}  + C$, where $C$ is the constant of integration, then the ordered point $\brak{\lambda, f\brak{\theta}}$ is:
    \begin{multicols}{2}
    \begin{enumerate}
        \item $\brak{-1, 1-\tan \theta}$\\
        \item $\brak{-1, 1+\tan \theta}$
        \item $\brak{1, 1+\tan \theta}$\\
        \item $\brak{1, 1-\tan \theta}$
    \end{enumerate}
    \end{multicols}

    \item Let $\Vec{a}$, $\vec{b}$, and $\vec{c}$ be three vectors such that $\abs{\overrightarrow a}=3$ , $\abs{\overrightarrow b}=5$ ,$\overrightarrow{a}\cdot \overrightarrow{b}= 10$ and the angle between $\overrightarrow{b}$ and $\overrightarrow{c}$ is $\frac{\pi}{3}$. If $\overrightarrow{a}$ is perpendicular to vector $\overrightarrow{b} \times \overrightarrow{c}$ ,  then $\abs{\overrightarrow{a} \times \brak{\overrightarrow{b}\times \overrightarrow{c}}}$ is equal to \rule{1cm}{0.15mm}\\

   \item If $\comb{}{r} = \comb{25}{r}$ and $\comb{}{0}+ 5 \cdot \comb{}{1}+ 9 \cdot \comb{}{2}+ \dots + 101 \cdot \comb{}{25} = 2^{25}. k $ then $k$ is equal to \rule{1cm}{0.15mm}\\


   \item If the curves $x^2-6x+y^2+8=0$ and $x^2-8y+y^2+16-k=0$ , $\brak{k>0}$ touch each other at a point , then the largest value of $k$ is \rule{1cm}{0.15mm}\\

   \item The number of terms common to the A.P.'s $3,7,11,\dots , 407$ and $2,9,16,\dots , 709$ is \rule{1cm}{0.15mm}\\


   \item If the distance between the plane , $23x-10y-2z+48=0$ and the plane containing the lines $\frac{x+1}{2} = \frac{y-3}{4} = \frac{z+1}{3}$ and $\frac{x+3}{2} = \frac{y+2}{6} = \frac{z-1}{\lambda}$ , $\brak{\lambda \in R}$ is equal to $\frac{k}{\sqrt{633}}$ the $k$ is equal to \rule{1cm}{0.15mm}\\
\end{enumerate}


\end{document}

